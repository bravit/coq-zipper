% В этом файле следует писать текст работы, разбивая его на
% разделы (section), подразделы (subsection) и, если нужно,
% главы (chapter).

% Предварительно следует указать необходимую информацию
% в файле SETUP.tex

%% В этот файл не предполагается вносить изменения

% В этом файле следует указать информацию о себе
% и выполняемой работе.

\documentclass [fontsize=14pt, paper=a4, pagesize, DIV=calc]%
{scrartcl}
% ВНИМАНИЕ! Для использования глав поменять
% scrartcl на scrreprt

% Здесь ничего не менять
\usepackage [T2A] {fontenc}   % Кириллица в PDF файле
\usepackage [utf8] {inputenc} % Кодировка текста: utf-8
\usepackage [russian] {babel} % Переносы, лигатуры

%%%%%%%%%%%%%%%%%%%%%%%%%%%%%%%%%%%%%%%%%%%%%%%%%%%%%%%%%%%%%%%%%%%%%%%%
% Создание макроса управления элементами, специфичными
% для вида работы (курс., бак., маг.)
% Здесь ничего не менять:
\usepackage{ifthen}
\newcounter{worktype}
\newcommand{\typeOfWork}[1]
{
	\setcounter{worktype}{#1}
}

% ВНИМАНИЕ!
% Укажите тип работы: 0 - курсовая, 1 - бак., 2 - маг.,
% 3 - бакалаврская с главами.
\typeOfWork{1}
% Считается, что курсовая и бак. бьются на разделы (section) и
% подразделы (subsection), а маг. — на главы (chapter), разделы и
%  подразделы. Если хочется,
% чтобы бак. была с главами (например, если она большая),
% надо выбрать опцию 3.

% Если при выборе 2 или 3 вы забудете поменять класс
% документа на scrreprt (см. выше, в самом начале),
% то получите ошибку:
% ./aux/appearance.tex:52: Package scrbase Error: unknown option ` chapterprefix=

%%%%%%%%%%%%%%%%%%%%%%%%%%%%%%%%%%%%%%%%%%%%%%%%%%%%%%%%%%%%%%%%%%%%%%%%
% Информация об авторе и работе для титульной страницы

\usepackage {titling}

% Имя автора в именительном падеже (для маг.)
\newcommand {\me}{%
И.\,И.~Иванов%
}

% Имя автора в родительном падеже (для курсовой и бак.)
\newcommand {\byme}{%
П.\,А.~Грахова%
}

% Научный руководитель
\newcommand{\supervisor}%
{ст.~п.~В. Н. Брагилевский}

% идентифицируем пол (только для курсовой и бак.)
\newcommand{\bystudent}{
Студента %Студентки % Для курсовой: с большой буквы
}

% Год публикации
\date{2019}

% Название работы
\title{Верификация структуры данных "зиппер"}

% Кафедра
%

\newcommand {\direction} {%
Направление подготовки\\01.03.02 ---
Прикладная математика\\и информатика%
}

%%%%%%%%%%%%%%%%%%%%%%%%%%%%%%%%%%%%%%%%%%%%%%%%%%%%%%%%%%%%%%%%%%%%%%%%
% Другие настраиваемые элементы текста

% Листинги с исходным кодом программ: укажите язык программирования
\usepackage{fontawesome}
\usepackage{listings,url}

% Изменять вид нумерации можно в начале нумерации:
% \begin{enumerate}[1)] (В квадратных скобках указан желаемый вид)
\usepackage[shortlabels]{enumitem}
                    \setlist[enumerate, 1]{1.}

% Гиперссылки: настройте внешний вид ссылок
\usepackage%
[pdftex,unicode,pdfborder={0 0 0},draft=false,%backref=page,
    hidelinks, % убрать, если хочется видеть ссылки: это
               % удобно в PDF файле, но не должно появиться на печати
    bookmarks=true,bookmarksnumbered=false,bookmarksopen=false]%
{hyperref}


\usepackage {amsmath}      % Больше математики
\usepackage {amssymb}
\usepackage {textcase}     % Преобразование к верхнему регистру
\usepackage {indentfirst}  % Красная строка первого абзаца в разделе

\usepackage {fancyvrb}     % Листинги: определяем своё окружение Verb
\DefineVerbatimEnvironment% с уменьшенным шрифтом
	{Verb}{Verbatim}
	{fontsize=\small}

% Вставка рисунков
\usepackage {graphicx}

% Общее оформление
% ----------------------------------------------------------------
% Настройка внешнего вида

%%% Шрифты

% если закомментировать всё — консервативная гарнитура Computer Modern
\usepackage{paratype} % профессиональные свободные шрифты
%\usepackage {droid}  % неплохие свободные шрифты от Google
%\usepackage{mathptmx}
%\usepackage {mmasym}
%\usepackage {psfonts}
%\usepackage{lmodern}
%var1: lh additions for bold concrete fonts
%\usepackage{lh-t2axccr}
%var2: the package below could be covered with fd-files
%\usepackage{lh-t2accr}
%\usepackage {pscyr}

% Геометрия текста

\usepackage{setspace}       % Межстрочный интервал
\onehalfspacing

\newlength\MyIndent
\setlength\MyIndent{1.25cm}
\setlength{\parindent}{\MyIndent} % Абзацный отступ
\frenchspacing            % Отключение лишних отступов после точек
\KOMAoptions{%
    DIV=calc,         % Пересчёт геометрии
    numbers=endperiod % точки после номеров разделов
}

                            % Консервативный вариант:
%\usepackage                % ручное задание геометрии
%[%                         % (не рекомендуется в проф. типографии)
%  margin = 2.5cm,
  %includefoot,
  %footskip = 1cm
%] %
%  {geometry}

%%% Заголовки


\ifthenelse{{\value{worktype} > 1}}{%
  \KOMAoptions{%
      headings=normal,   % размеры заголовков поменьше стандартных
      chapterprefix=true,% Печатать слово Глава
      appendixprefix=true% Печатать слово Приложение
  }
}{% Печатать слово Приложение даже если нет глав
  \newcommand*{\appendixmore}{%
    \renewcommand*{\sectionformat}{%
    \appendixname~\thesection\autodot\enskip}
    \renewcommand*{\sectionmarkformat}{%
      \appendixname~\thesection\autodot\enskip}
  }
}

% шрифт для оформления глав и названия содержания
\newcommand{\SuperFont}{\Large\sffamily\bfseries}

% Заголовок главы
\ifthenelse{\value{worktype} > 1}{%
\renewcommand{\SuperFont}{\Large\normalfont\sffamily}
\newcommand{\CentSuperFont}{\centering\SuperFont}
\usepackage{fncychap}
\ChNameVar{\SuperFont}
\ChNumVar{\CentSuperFont}
\ChTitleVar{\CentSuperFont}
\ChNameUpperCase
\ChTitleUpperCase
}

% Заголовок (под)раздела с абзацного отступа
\addtokomafont{sectioning}{\hspace{\MyIndent}}

\renewcommand*{\captionformat}{~---~}
\renewcommand*{\figureformat}{Рисунок~\thefigure}

% Плавающие листинги
\usepackage{float}
\floatstyle{ruled}
\floatname{ListingEnv}{Листинг}
\newfloat{ListingEnv}{htbp}{lol}[section]

% точка после номера листинга
\makeatletter
\renewcommand\floatc@ruled[2]{{\@fs@cfont #1.} #2\par}
\makeatother


%%% Оглавление
\usepackage{tocloft}

% шрифт и положение заголовка
\ifthenelse{\value{worktype} > 1}{%
\renewcommand{\cfttoctitlefont}{\hfil\SuperFont\MakeUppercase}
}{
\renewcommand{\cfttoctitlefont}{\hfil\SuperFont}
}

% слово Глава
\usepackage{calc}
\ifthenelse{\value{worktype} > 1}{%
\renewcommand{\cftchappresnum}{Глава }
\addtolength{\cftchapnumwidth}{\widthof{Глава }}
}

% Очищаем оформление названий старших элементов в оглавлении
\ifthenelse{\value{worktype} > 1}{%
\renewcommand{\cftchapfont}{}
\renewcommand{\cftchappagefont}{}
}{
\renewcommand{\cftsecfont}{}
\renewcommand{\cftsecpagefont}{}
}

% Точки после верхних элементов оглавления
\renewcommand{\cftsecdotsep}{\cftdotsep}
%\newcommand{\cftchapdotsep}{\cftdotsep}

\ifthenelse{\value{worktype} > 1}{%
    \renewcommand{\cftchapaftersnum}{.}
}{}
\renewcommand{\cftsecaftersnum}{.}
\renewcommand{\cftsubsecaftersnum}{.}
\renewcommand{\cftsubsubsecaftersnum}{.}

%%% Списки (enumitem)

\usepackage {enumitem}      % Списки с настройкой отступов
\setlist %
{ %
  leftmargin = \parindent, itemsep=.5ex, topsep=.4ex
} %

% По ГОСТу нумерация должны быть буквами: а, б...
%\makeatletter
%    \AddEnumerateCounter{\asbuk}{\@asbuk}{м)}
%\makeatother
%\renewcommand{\labelenumi}{\asbuk{enumi})}
%\renewcommand{\labelenumii}{\arabic{enumii})}

%%% Таблицы: выбрать более подходящие

\usepackage{booktabs} % считаются наиболее профессионально выполненными
%\usepackage{ltablex}
%\newcolumntype {L} {>{---}l}

%%% Библиография

\usepackage{csquotes}        % Оформление списка литературы
\usepackage[
  backend=biber,
  hyperref=auto,
  sorting=none, % сортировка в порядке встречаемости ссылок
  language=auto,
  citestyle=gost-numeric,
  bibstyle=gost-numeric
]{biblatex}
\addbibresource{biblio.bib} % Файл с лит.источниками

% Настройка величины отступа в списке
\ifthenelse{\value{worktype} < 2}{%
\defbibenvironment{bibliography}
  {\list
     {\printtext[labelnumberwidth]{%
    \printfield{prefixnumber}%
    \printfield{labelnumber}}}
     {\setlength{\labelwidth}{\labelnumberwidth}%
      \setlength{\leftmargin}{\labelwidth}%
      \setlength{\labelsep}{\dimexpr\MyIndent-\labelwidth\relax}% <----- default is \biblabelsep
      \addtolength{\leftmargin}{\labelsep}%
      \setlength{\itemsep}{\bibitemsep}%
      \setlength{\parsep}{\bibparsep}}%
      \renewcommand*{\makelabel}[1]{\hss##1}}
  {\endlist}
  {\item}
}{}

% ----------------------------------------------------------------
% Настройка переносов и разрывов страниц

\binoppenalty = 10000      % Запрет переносов строк в формулах
\relpenalty = 10000        %

\sloppy                    % Не выходить за границы бокса
%\tolerance = 400          % или более точно
\clubpenalty = 10000       % Запрет разрывов страниц после первой
\widowpenalty = 10000      % и перед предпоследней строкой абзаца

% ----------------------------


% Стили для окружений типа Определение, Теорема...
% Оформление теорем (ntheorem)

\usepackage [thmmarks, amsmath] {ntheorem}
\theorempreskipamount 0.6cm

\theoremstyle {plain} %
\theoremheaderfont {\normalfont \bfseries} %
\theorembodyfont {\slshape} %
\theoremsymbol {\ensuremath {_\Box}} %
\theoremseparator {:} %
\newtheorem {mystatement} {Утверждение} [section] %
\newtheorem {mylemma} {Лемма} [section] %
\newtheorem {mycorollary} {Следствие} [section] %

\theoremstyle {nonumberplain} %
\theoremseparator {.} %
\theoremsymbol {\ensuremath {_\diamondsuit}} %
\newtheorem {mydefinition} {Определение} %

\theoremstyle {plain} %
\theoremheaderfont {\normalfont \bfseries} 
\theorembodyfont {\normalfont} 
%\theoremsymbol {\ensuremath {_\Box}} %
\theoremseparator {.} %
\newtheorem {mytask} {Задача} [section]%
\renewcommand{\themytask}{\arabic{mytask}}

\theoremheaderfont {\scshape} %
\theorembodyfont {\upshape} %
\theoremstyle {nonumberplain} %
\theoremseparator {} %
\theoremsymbol {\rule {1ex} {1ex}} %
\newtheorem {myproof} {Доказательство} %

\theorembodyfont {\upshape} %
%\theoremindent 0.5cm
\theoremstyle {nonumberbreak} \theoremseparator {\\} %
\theoremsymbol {\ensuremath {\ast}} %
\newtheorem {myexample} {Пример} %
\newtheorem {myexamples} {Примеры} %

\theoremheaderfont {\itshape} %
\theorembodyfont {\upshape} %
\theoremstyle {nonumberplain} %
\theoremseparator {:} %
\theoremsymbol {\ensuremath {_\triangle}} %
\newtheorem {myremark} {Замечание} %
\theoremstyle {nonumberbreak} %
\newtheorem {myremarks} {Замечания} %


% Титульный лист
% Макросы настройки титульной страницы
% В этот файл не предполагается вносить изменения

%\usepackage {showframe}

% Вертикальные отступы на титульной странице
\newcommand{\vgap}{\vspace{16pt}}

% Помещение города и даты в нижний колонтитул
\usepackage{scrlayer}
\DeclareNewLayer[
  foot,
  foreground,
  contents={%
    \raisebox{\dp\strutbox}[\layerheight][0pt]{%
      \parbox[b]{\layerwidth}{\centering Ростов-на-Дону\\ \thedate%
       \\\mbox{}
       }}%
  }
]{titlepage.foot.fg}
\DeclareNewPageStyleByLayers{titlepage}{titlepage.foot.fg}


\AtBeginDocument %
{ %
  %
  \begin{titlepage}
  %
    \thispagestyle{titlepage}

    {\centering
    %
    \MakeTextUppercase {МИНИСТЕРСТВО ОБРАЗОВАНИЯ И НАУКИ РФ}

    \vgap

    Федеральное государственное автономное образовательное\\
    учреждение высшего образования\\
    \MakeTextUppercase {Южный федеральный университет}

    \vgap

	Институт математики, механики и компьютерных наук
    имени~И.\,И.\,Воровича

    \vgap

    \direction

    \vspace* {\fill}

    \ifthenelse{\value{worktype} = 2}{%
    \me

    \vgap}{}

    {\usefont{T2A}{PTSansCaption-TLF}{m}{n}
    \MakeTextUppercase{\thetitle}}

    \ifthenelse{\value{worktype} = 2}{%
     \vgap

    Магистерская диссертация}{}
    \ifthenelse{\value{worktype} = 0}{
     \vgap

    Курсовая работа
    }{}%
    \ifthenelse{\value{worktype} = 1 \OR \value{worktype} = 3}{
     \vgap

    Выпускная квалификационная работа\\
    на степень бакалавра
    }{}%

    \vspace {\fill}

    \begin{flushright}
    \ifthenelse{\value{worktype} = 0 \OR 
                \value{worktype} = 1 \OR
                \value{worktype} = 3}{
      \bystudent \ifthenelse{\value{worktype} = 0}{3}{4}\ курса\\
      \byme
    }{}

    \vgap

    Научный руководитель:\\
    \supervisor\\
    \ifthenelse{\value{worktype} = 2}{%
    Рецензент:\\
    ученая степень, ученое звание, должность
    И. О. Фамилия
    }{}
	\end{flushright}
\ifthenelse{\value{worktype} = 0}{
\vspace{\fill}
        \begin{flushleft}
          \begin{tabular}{cc}
            \underline{\hspace{4cm}}&\underline{\hspace{5cm}}\\
            {\small оценка (рейтинг)} & {\small  подпись руководителя}\\
          \end{tabular}
          \\[1cm]
        \end{flushleft}
}{}
\ifthenelse{\value{worktype} = 1 \OR \value{worktype} = 3}{
\vspace{\fill}
        \begin{flushleft}
Допущено к защите:\\руководитель направления
\underline{\hspace{4cm}}
В.\,С.\,Пилиди
        \end{flushleft}
}{}


  	\vspace {\fill}
  %Ростов-на-Дону

    %\thedate

  }\end{titlepage}
  %
  %
  \tableofcontents
  %
  \clearpage
} %


% Команды для использования в тексте работы


% макросы для начала введения и заключения
\newcommand{\Intro}{\addsec{Введение}}
\ifthenelse{\value{worktype} > 1}{%
    \renewcommand{\Intro}{\addchap{Введение}}%
}

\newcommand{\Conc}{\addsec{Заключение}}
\ifthenelse{\value{worktype} > 1}{%
    \renewcommand{\Conc}{\addchap{Заключение}}%
}

% Правильные значки для нестрогих неравенств и пустого множества
\renewcommand {\le} {\leqslant}
\renewcommand {\ge} {\geqslant}
\renewcommand {\emptyset} {\varnothing}

% N ажурное: натуральные числа
\newcommand {\N} {\ensuremath{\mathbb N}}

% значок С++ — используйте команду \cpp
\newcommand{\cpp}{%
C\nolinebreak\hspace{-.05em}%
\raisebox{.2ex}{+}\nolinebreak\hspace{-.10em}%
\raisebox{.2ex}{+}%
}

% Неразрывный дефис, который допускает перенос внутри слов,
% типа жёлто-синий: нужно писать жёлто"/синий.
\makeatletter
    \defineshorthand[russian]{"/}{\mbox{-}\bbl@allowhyphens}
\makeatother


\endinput

% Конец файла


\NewBibliographyString{langjapanese}
\NewBibliographyString{fromjapanese}

\newcommand{\tbttr}{BT\_tr}
\newcommand{\tcoq}{\texttt{Coq}~}
\newcommand{\tqc}{\texttt{QuickChick}~}

\begin{document}

\Intro

В процессе разработки программного обеспечения неизбежно возникает вопрос, удовлетворяет ли написанный программистом код некоторым заранее заданным спецификациям, согласно которым должны были быть разработаны алгоритмы? Этот процесс проверки корректности кода, или, иначе, верификации, может быть построен как с использованием так называемых unit-тестов, проверяющих код на большом количестве разнообразных входных данных, так и с помощью формального доказательства корректности интересующих нас участков кода. Несмотря на то, что второй вариант тестирования может формально гарантировать абсолютную корректность, в силу своей сложности он используется сравнительно редко, в промышленности куда более популярен первый способ. Однако в последнее время, с развитием систем автоматической верификации теорем, приобретает популярность использование комбинации этих двух методов, когда некоторые критически важные участки кода либо алгоритмы формулируются в виде теорем и верифицируются формально в подобных средах, в то время как корректность второстепенных частей "доказывается" посредством прогонки на тестах, возможно, в этих же средах, с помощью некоторых их расширений.

Одной из самых популярных систем автоматической верификации теорем является система Coq, разработанная в INRIA. В этой среде для тестирования написанных определений и теорем может использоваться плагин QuickChick, позволяющий сгенерировать набор случайных входных данных и проверить, является ли интересующая нас теорема либо свойство верными на этом наборе. Подобная проверка не гарантирует абсолютной правильности проверенных теорем (то есть то, что они будут верны для любых входных данных), но вероятность этого весьма высока, если был успешно пройден подобный unit-тест, что может вдохновить программиста либо исследователя на поиск настоящего доказательства.

В данной работе ставится цель исследовать одну структуру данных с позиций подобного стиля верификации. Более точно, исследуется вариант структуры данных "Зиппер" для обобщенных деревьев, описывается сама структура данных и набор функций для работы с ней, формулируются желаемые свойства, которым должна подобная структура данных удовлетворять, составляются тесты и доказывается корректность подобных свойств в нашей формулировке. Похожий процесс при разработке ПО выполняется часто, цель данной работы --- показать, что Coq может являться удобной средой для подобных манипуляций, когда важен высокий уровень доверия к полученным алгоритмам.

\section{Структура данных Зиппер}

\subsection{Зиппер в общем смысле}
Строго говоря, Zipper/Зиппер --- не есть структура данных в том смысле, который в него вкладывается исследователями и специалистами в области компьютерных наук, но конструкция, позволяющая на базе некоторой сложной структуры данных построить новую структуру, позволяющую при работе с данными обращаться лишь к той части исходных данных, которые требуют модификации либо обращения, оставляя оставшуюся часть нетронутой. Особенно это актуально при реализации алгоритмов на чисто функциональных языках программирования, где прямая работа с интересующей нас частью данных может быть сложной задачей. 

Как пример, Зиппер для бинарного дерева может быть построен следующим образом: что-то там

\subsection{Зиппер для обобщенного дерева}

Зиппер был описан Gerard Huet ~\autocite{Huet1997} на примере обобщенного дерева, приведем здесь ту часть его описания, которая будет интересовать нас.

что-то там

\subsection{Описание в Coq}

Важно корректно описать изначальные структуры на внутреннем языке \tcoq, чтобы в дальнейшем избежать проблем с формальным доказательством интересующих нас свойств. Функциональная реализация алгоритмов и структур данных в этой работе базируется в основном на идеях Окасаки ~\autocite{Okasaki1996}.

Тип, описывающий Зиппер, может быть рассмотрен как произведение типов ``дерево'' и ``список контекстов'', где ``контекст'', в свою очередь, тоже тип, описанный выше. Возможности стандартной библиотеки \tcoq~ позволяют описать Зиппер как модуль, который можно затем использовать повторно

\begin{lstlisting}
Module TreeZipper (Import T: Typ).

Definition A := T.t.

Inductive Tree : Type :=
  | T_nil: Tree
  | T_tr:  A -> list Tree -> Tree.

Inductive Context : Type :=
  | T_move: nat -> A -> list Tree -> Context.

Definition ZipperTree := prod Tree (list Context).
...
\end{lstlisting}
что-то там про алгоритмы

\section{Тестирование с использованием QuickChick}

Как уже было сказано, есть возможность проводить тесты алгоритмов в самом \tcoq~ на случайных данных, используя \tqc.

\subsection{Общие замечания}

Ограничения, которые мы накладываем на Зиппер (например, контекст не может быть вида $cnt n, A, [
]$, не позволяют нам использовать стандартные механизмы \tqc, автоматически выводящие генераторы для заданных типов. В связи с этим стоит детальнее рассмотреть то, как можно вручную описать генераторы для интересующих нас структур.

Самым важным (и сложным) является описание функции, возвращающей генератор случайного дерева высоты $sz$. Функция работает следующим образом: в случае $sz  = 0$ возвращается лист (nil), иначе генерируется случайная величина типа \texttt{A}, затем рекурсивно строится список деревьев высоты $sz - 1$ и из полученных данных составляется исходное дерево.

\begin{lstlisting}
Fixpoint genTreeSized (sz: nat) (g: G A) : G (Tree) :=
  match sz with
  | 0 => returnGen T_nil
  | S sz' =>
    freq [ (1, returnGen T_nil) ;
      (sz, bindGen g (fun x =>
           bindGen (@listOf (Tree) (genTreeSized sz' g)) (fun l =>
         returnGen (T_tr x l)))) ]
  end.
\end{lstlisting}

Проблема в том, что, так как у нас имеется лишь два варианта выбора ( $sz = 0$ или $sz > 0$ ), то \tqc будет генерировать лист в примерно 50\% случаев, что не есть желаемый результат. В связи с этим используется стандартный комбинатор \texttt{freq}, позволяющий снизить вероятность появления nil до $\frac{1}{sz + 1}$.

что-то про Context

\subsection{Описание тестов}

Опять же, в силу фундаментальных ограничений \tqc, нельзя напрямую проверить сформулированные в \tcoq леммы на случайных данных. Требуется их переформулировать в терминах, доступных для проверки.

Описание пары тестов

\section{Формальная верификация с использованием Coq}

\subsection{Формулировка желаемых свойств}

Модифицируя или обозревая элементы дерева с использованием Зиппер, мы неявно предполагаем, что модификация будет затрагивать лишь ту часть дерева, которая не содержится в контексте (поскольку это есть определение Зиппер), и что некоторые свойства дерева, его ``целостность'', не будут нарушаться при выполнении стандартных операций. Было бы неплохо убедиться в том, что наши ожидания действительно будут выполняться.

Пусть у нас имеется дерево $T$ и некоторый предикат $P$ над деревом. Тогда будем говорить, что $P$ применимо к $T$, если $P$ верно для $T$ и всех его поддеревьев. Подобное обобщение хорошо тем, что позволяет при необходимости определить сколь угодно сложную функцию проверки ``правильности'' дерева. Это определение ``применимости'' может быть сформулировано в \tcoq следующим образом:

\begin{lstlisting}
Inductive PropertyOverTree : (Tree -> Prop) -> Tree -> Prop :=
  | P_nil: forall (P: Tree -> Prop),
    P T_nil -> PropertyOverTree P T_nil
  | P_tree: forall (P: Tree -> Prop) T,
    P T -> (forall T', IsSubtreeOf T' T -> P T') ->
    PropertyOverTree P T.
\end{lstlisting}

\Conc

% Печать списка литературы (библиографии)
\printbibliography[%{}
    heading=bibintoc%
    %,title=Библиография % если хочется это слово
]
% Файл со списком литературы: biblio.bib
% Подробно по оформлению библиографии:
% см. документацию к пакету biblatex-gost
% http://ctan.mirrorcatalogs.com/macros/latex/exptl/biblatex-contrib/biblatex-gost/doc/biblatex-gost.pdf
% и огромное количество примеров там же:
% http://mirror.macomnet.net/pub/CTAN/macros/latex/contrib/biblatex-contrib/biblatex-gost/doc/biblatex-gost-examples.pdf

\appendix
\ifthenelse{\value{worktype} > 1}{%
  \addtocontents{toc}{%
      \protect\renewcommand{\protect\cftchappresnum}{\appendixname\space}%
      \protect\addtolength{\protect\cftchapnumwidth}{\widthof{\appendixname\space{}} - \widthof{Глава }}%
  }%
}{
  \addtocontents{toc}{%
      \protect\renewcommand{\protect\cftsecpresnum}{\appendixname\space}%
      \protect\addtolength{\protect\cftsecnumwidth}{\widthof{\appendixname\space{}}}%
  }%
}

\end{document}